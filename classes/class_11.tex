\section{Class 11}

\begin{definition}
    (Change of basis matrix)
    \begin{align*}
        C_{S \to T} = \begin{pmatrix} 
        c_{11}  & c_{12} & \hdots & c_{1n} \\
        c_{21}  & c_{22} & \hdots & c_{2n} \\
        \vdots & \vdots \ddots & \vdots \\
        c_{n1} & c_{n2} & \hdots & c_{nn}
        \end{pmatrix}
    \end{align*} 
    Where the $i$-th column is the coordinates of $s_i$ with respect to basis $T$, is called the \textbf{basis change matrix} from $S$ to $T$
\end{definition}

\begin{remark}
    If $v= \sum\limits_{ j = 1}^{n} a_j s_j = \sum\limits_{i = 1}^{n}  b_i t_i$ then 
    \[
        [v]_T = C_{S \to T} [v]_S
    \]
    Similarly, if $C_{T \to S} = [d_{ij}]$  where 
    \[
        t_j = \sum\limits_{i = 1}^{n} d_{ij}s_i
    \]
    then 
    \[
        [v]_S = C_{T \to S} [v]_T
    \]

    Therefore, the proposition from Class 10 can be rephrased as 
    \[
        [v]_T = C_{S \to T} [v]_S, [v]_S = C_{ T \to S} [v]_T
    \]
    and 
    \[
        C_{S \to T} C_{T \to S} = I = C_{T \to S} C_{S \to T}
    \]
\end{remark}

\subsection{Representation of linear maps}

\begin{definition} 
    (Matrix representation of linear maps) \\

    Let $V, W$ be $F$-vector spaces, $S = (s_1, s_2, \hdots s_n)$ basis for $V$. $T = (t_1, t_2, \hdots t_m)$ basis for $W$. Let $\phi:V \to W$ linear. \\k

    There are uniquely determined coefficients $d_{ij} \in F$ such that 
    \[
        \phi(s_j ) = \sum\limits_{i = 1}^{m} d_{ij } t_i
    \]
    for all $1 \leq j \leq n$. \\

    The matrix 
    \[
        [\phi]_{S \to T} = [d_{ij}]_{1 \leq i \leq m, 1 \leq j \leq n}
    \]

    is the $m \times n$ matrix representing $\phi$ with respect to bases $S$ and $T$.
\end{definition}

\begin{remark}          
    If $\phi = Id_{V}: V \to V, v \mapsto v$, and $S, T$ bases for $F$
    \[
        [Id_V]_{S \to T} = C_{S \to T}
    \]
\end{remark}            

\begin{proposition}
    Let $V, W$ be $F$-vector spaces.  \\

    Let $[v]_S = \gamma_S (v)$ be the coordinate representation of $v$ with respect to $S$. \\
    Let $[\phi(v)]_T = \gamma_T (\phi(v))$ be the coordinate representation of $\phi(v)$ with respect to $T$, then 
    \[
        \left[ \phi(v) \right]_T = \left[ \phi \right]_{S \to T} \left[ v \right]_S
    \]
\end{proposition}
\begin{proof}
    Let 
    \[
        v = \sum\limits_{i = 1}^{n} a_j s_j \in V.
    \]
    Let $d_{ij}$ be defined by 
    \[
        \phi(s_j) = \sum\limits_{i = 1}^{m} d_{ij}t_i
    \]
    then
    \begin{align*}
        \phi(v) &= \sum\limits_{j = 1}^{n}  a_j \phi(s_j) \\
        &= \sum\limits_{j = 1}^{n}  a_j \sum\limits_{i = 1}^{m} d_{ij} t_i \\
        &= \sum\limits_{i = 1}^{m} \left(  \sum\limits_{ j = 1}^{n} d_{ij}a_j \right)  t_i
    \end{align*}

    Therefore 
    \begin{align*}
        \left[  \phi(v)\right]_T &= \begin{bmatrix}
            \sum\limits_{j = 1}^{n} d_{ij}a_j \\
            \sum\limits_{j = 1}^{n} d_{2j}a_j \\
            \vdots \\
            \sum\limits_{j = 1}^{n} d_{mj}a_j \\
        \end{bmatrix} \\
        &= \left[ d_{ij} \right] \begin{bmatrix} 
            a_1 \\ a_2 \\ \vdots \\ a_n
        \end{bmatrix} \\ 
        &= \left[ \phi \right]_{S \to T} \left[ v \right]_S
    \end{align*}
\end{proof}

\begin{theorem}
    Let $V, W$ be $F$-vector spaces with $S = (s_1, s_2, \hdots s_n)$, $T = (t_1, t_2, \hdots t_n)$ bases respectively. \\

    The map 
    \begin{align*}
        D_{S \to T} : Hom_F(V, W) &\to F^{m \times n} \\
        \phi & \mapsto D_{S \to T} (\phi) = \left[ \phi \right]_{S \to T}
    \end{align*}
    is an isomorphism of $F$-vector spaces.
\end{theorem}

\begin{proof} 
    We want to show that $D_{S \to T}$ is linear and bijective. \\

    \textbf{(1) Linearity}:  \\
    Let $\phi, \psi \in Hom_F(V, W)$  and $c \in F$. \\

    Let $a_{ij}$ such that $\phi(s_j) = \sum\limits_{i = 1}^{n} a_{ij} t_i$. Let $b_{ij}$ such that $\psi(s_j) = \sum\limits_{i = 1}^{n} b_{ij} t_i$ \\

    Then, 
    \begin{align*}
        \left( \phi + c \psi \right)  (s_j) &= \phi(s_j) + c \psi(s_j) \\
        &= \sum\limits_{i = 1}^{n}  (a_{ij} + cb_{ij}) t_i
    \end{align*}
    Hence 
    \begin{align*}
        \left[ \phi + c \psi \right]_{S \to T} &= \left[ \left( a_{ij }  + cb_{ij}\right)  \right]_{ij} \\
        &= \left[ a_{ij}  \right]  + c \left[ b_{ij}  \right] \\
        &= \left[ \phi \right]_{S \to T} + c \left[ \psi \right]_{S \to T}
    \end{align*}
    Hence $D_{S \to T}$ is linear. \\

    \textbf{(2): Injectivity}
    $D_{S \to T}$ is injective because if $ \phi, \psi \in Hom_F(V, W)$ , such that 
    \[
        \left[ \phi \right]_{S \to T} = \left[ \psi \right]_{S \to T}
    \]
    Then
    \[
        \phi(s_j) = \psi(s_j)
    \]

    Since $S$ is a basis and $\phi, \psi$ linear, this implies $\phi = \psi$ \\

    \textbf{(3) Surjectivity}: Given $A = \left[ a_{ij} \right] \in F^{m \times n}$, the isomorphism theorem guarantees the existence of a $\varphi$
    \[
        \varphi: V \to W, \varphi(s_j) = w_j
    \]

    Where
    \[
        w_j = \sum\limits_{i = 1}^{n} a_{ij}   t_i
    \]
    And 
    \[
        \left[ \varphi \right]_{S \to T} = A
    \]
\end{proof}

\begin{remark}
    By the theorem above, 
    \[
        \left( D_{S \to T} \right)^{-1} : F^{m \times n} \to Hom_F(V, W)
    \]
    is isomorphism.
\end{remark}

\begin{remark}
    Let $E^{kl} \in F^{m \times n}$ be the matrix that has all entries zero except at the $(k, l)$ entry for $1 \leq k \leq m, 1 \leq l \leq n$.  \\

    By isomorphism theorem, 
    \[
        \{ (D_{S \to T})^{-1} (E_{kl}) \}_{1 \leq k \leq m,\ 1 \leq l \leq n}
    \]
    is a basis for $Hom_F(V, W)$.
\end{remark}

\begin{corollary}
    if $\dim V = n, \dim W = m$,, then 
    \[
        \dim \left( Hom_F(V, W) \right) = mn
    \]
\end{corollary}
\begin{remark}
    \[
    \left( D_{S \to T} \right) (E_{kl}) (s_j) = \begin{cases}
        0 \text{ if } j \neq l \\
        s_k \text{ if } j = l
    \end{cases}
    \]
\end{remark}

\begin{theorem}
    Let $V, W, X$ have basis $S, T, U$ respectively. Let 
    \[
        \phi \in Hom_F(V, W), \psi \in Hom_F(W, X)
    \]
    Then 
    \[
        \left[ \psi \circ \phi \right]_{S \to U} = \left[ \psi \right]_{T \to U} \left[ \phi \right]_{S \to T}
    \]
\end{theorem}

\begin{proof}
    Suppose 
    \begin{align*}
        S &= \left( s_1, s_2, \hdots s_n \right)  \\
        T &= \left( t_1, t_2, \hdots t_m \right)  \\
        U &= \left( u_1, u_2, \hdots u_l \right)
    \end{align*}

    Let $\phi(s_j) = \sum\limits_{i=1}^{n} a_{ij} t_i $, so that 
    \[
        \left[ \phi \right]_{S \to T} = \left[ a_{ij} \right] 
    \]
    Let $\psi(t_j) = \sum\limits_{i = 1}^{l} b_{ij} u_i$, so that 
    \[
        \left[ \psi \right]_{T \to U} = \left[ b_{ij} \right] 
    \]

    \begin{align*}
        \left( \psi \circ \phi \right) (s_j) &= \psi \left( \phi(s_j) \right)  \\
        &= \psi \left( \sum\limits_{k = 1}^{m} a_{kj} t_k  \right)  \\
        &= \sum\limits_{k = 1}^{m}  a_{kj} \psi \left( t_k \right)  \\
        &= \sum\limits_{k= 1}^{m}  a_{kj} \left( \sum\limits_{i = 1}^{l}  b_{ik} u_i\right)  \\
        % &= \psi \circ \phi(s_j) \\
        &= \sum\limits_{i = 1}^{l} \left( \sum\limits_{k = 1}^{m} b_{ik} a_{kj} \right) u_i
    \end{align*} 
    Hence, by definition of matrix multiplication
    \[
        \left[ \psi \circ \phi \right]_{S \to U} = = \left[ \sum\limits_{k = 1}^{m} b_{ik} a_{kj}\right]_{ij}  \left[ \psi \right] \left[ \phi \right] 
    \]
\end{proof}

\begin{corollary}
    Let $V$ be a $F$-vector space with bases $S, \tilde{S}$. Let $W$ be a $F$-vector space with bases $T \tilde{T}$. Let $\phi : V \to W$ linear.  

    Then 
    \[
        \left[ \phi \right]_{\tilde{S} \to \tilde{T}} = C_{T \to \tilde{T}} \left[ \phi \right]_{S \to T} C_{ \tilde{S} \to S}
    \]
\end{corollary}

\begin{proof}
    \begin{align*}
        & C_{T \to \tilde{T}} \left[ \phi \right]_{S \to T} C_{\tilde{S} \to S} \\
        =& \left[ Id_W \right]_{T \to \tilde{T}} \left[ \phi \right]_{S \to T} \left[ Id_V \right]_{\tilde{S} \to S} \\
        =& \left[ Id_W \circ \phi \circ Id_V \right]_{\tilde{S} \to \tilde{T}} \\
        =& \left[ \phi \right]_{\tilde{S} \to \tilde{T}}
    \end{align*}
\end{proof}

\begin{remark}
    Say $\dim V = n$ and $S$ is a basis for $V$. 
    
    $End_F(V)$ is an $F$-algebra with composition as multiplication. $Mat_n(F)$ is also al $F$-algebra with matrix multiplication as multiplication. 
\end{remark}
\begin{proof}
    From theorem, 
    \begin{align*}
        D_S : End_n(V) &\to Mat_n(F) \\
         \phi &\mapsto \left[ \phi \right]_{S \to S}
    \end{align*}
    is an isomorphism of $F$-vector spaces.  \\

    The above theorem says that 
    \[
        D_S \left( \psi \circ \phi \right)  = D_S \left( \psi \right)  \cdot D_S \left(  \phi \right) 
    \]
    and $D_S$ is an isomorphism of $F$-algebra. 
\end{proof}
\begin{remark}
    Say $\dim V = n$ and $S$ is a basis for $V$. 
    \begin{align*}
        D_S: Gl(V) &\to Gl(F) = \{ A \in Mat_n(F): A \text{ invertible} \}  \\
        \phi &\mapsto D_s( \phi) = \left[ \phi \right]_{S \to S}
    \end{align*}
    $D_S$ is a group isomorphism.
\end{remark}























\newpage