\section{Class 1}

\subsection{Fields}

\begin{definition}
(Field): A field $F$ is a set with two binary operations

\[
+: F \times F \to F,\ (x, y) \mapsto x + y
\]

\[
\cdot: F \times F \to F,\ (x, y) \mapsto x \cdot y
\]

that satisfy these properties:
\begin{itemize}
\item (A0) existence of additive identity or neutral element: there is $0 \in F$ such that $x + 0 = x$ for all $x \in F$
\item (A1) additive commutativity: for all $x, y \in F$, $x + y = y + x$
\item (A2) additive associativity: for all $x, y, z \in F$, $x + (y + z) = (x + y) + z$
\item (A3) existence of additive inverse: for all $x \in F$ there is $y$ such that $x + y = 0$
\item (M0) existence of multiplicative identity or neutral element: there is $1 \in F, 1 \neq 0$ such that $x \cdot 1 = 1 \cdot x = x$ for all $x$
\item (M1) multiplicative commutativity: for all $x, y \in F$, $x \cdot y = y \cdot x$
\item (M2) multiplicative associativity: for all $x, y, z \in F$, $x \cdot (y \cdot z) = (x \cdot y) \cdot z$
\item (M3) existence of multiplicative inverse: for all $x \in F, x \neq 0$ there is $y$ such that $x \cdot y = 1$
\item (D) distributivity: for all $x, y, z \in F$, $(x + y) \cdot z = x \cdot z + y \cdot z$
\end{itemize}
\end{definition}

\begin{remark}
$\{ 0 \}$ is not a field because we require that the multiplicative identity be distinct from 0. If we allowed $0 = 1$, then $F$ is the trivial field, i.e., $F = \{ 0 \}$. \\
\end{remark}

\begin{remark}
The smallest field is $F_2 = \{ 0, 1 \}$ with addition and multiplication defined as: \\
\begin{center}
    \begin{tabular}{c | c | c}
        $+$ & $0$ & $1$ \\
        \hline
        $0$ & $0$ & $1$ \\
        $1$ & $1$ & $0$ \\
    \end{tabular}
\end{center}
\begin{center}
    \begin{tabular}{c | c | c}
        $\cdot$ & $0$ & $1$ \\
        \hline
        $0$ & $0$ & $0$ \\
        $1$ & $0$ & $1$ \\
    \end{tabular}
\end{center}
\end{remark}

\begin{remark}
If $(F, +, \cdot)$ is a field, then $0 \cdot x = 0$ for all $x$.  
\end{remark}

\begin{proof}
\textbf{Proof}
\[
0 \cdot z = (0 + 0) \cdot z = 0 \cdot z + 0 \cdot z
\]

Adding the additive inverse of $0 \cdot z$ to both sides, we get
\[
0 = 0 \cdot z
\]
\end{proof}

\begin{remark}
The additive and multiplicative inverses are unique.
\end{remark}

\begin{proof}
Let $x \in F$, suppose $y, z$ are both additive inverses of $x$.
\begin{align*}
    y &= y \\
    y &= y + 0 \\
    y &= y + (x + z) \\
    y &= (y + x) + z \\
    y &= z
\end{align*}
\end{proof}

\begin{remark}
Since the additive and multiplicative inverses are unique, we denote the additive inverse and multiplicative inverse of $x$ respectively as $-x$ and $x^{-1}$.
\end{remark}

\begin{definition}
(Group):  A set $G$ with a binary operation $*$ is a group if it has
\begin{itemize}
\item existence of inverse
\item existence of identity
\item associativity
\end{itemize}
\end{definition}

\begin{remark}
Note that commutativity is not required. A group with commutativity is known as a \textbf{commutative group}.
\end{remark}

\begin{definition}
(Field): $(F, +, \cdot)$ is a field if
\begin{itemize}
    \item $(F, +)$ is a commutative group
    \item $(F \setminus \{ 0 \}, \cdot )$ is a commutative group
    \item distributive properties hold
\end{itemize} 
\end{definition}

\newpage








