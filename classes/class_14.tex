\section{Class 14}

\subsection{Quotients, cont'd}

Recall from last time (homomorphism theorem) that if $\varphi: V \to W$ is a linear map between F-vector spaces, then 
\[
    \tilde{\varphi}: V/{\ker \varphi} \to Im \varphi, [v] \mapsto \varphi(v)
\]

is well defined isomorphism. 

\begin{corollary}
    Every linear map $\varphi: V \to W$ factors as 
    \[
        \varphi = i \circ \overline{\varphi} \circ \pi
    \]
    where 
    \begin{itemize}
        \item $\pi: V \to V/ \ker \varphi$ is the canonical projection 
        \item $i: Im \varphi \to W$ is the inclusion map
        \item $\overline{\varphi}: V / \ker \varphi \to Im\phi$ is isomorphism
    \end{itemize} 
\end{corollary}

\begin{proposition}
    (Dimension of a quotient space) Let $V$ be a finite dimensional vector space over $F$, and let $W \leq V$, then 
    \[
        \dim \left( V /W \right)  = \dim V - \dim W
    \]
\end{proposition}

\begin{proof}
    Say $\dim W = m$. Take $(w_1, \hdots, w_m)$ basis for $W$. Extend it to a basis of $V$, $S = (w_1, w_2, \hdots w_m, v_{m + 1}, v_{m + 2}, \hdots v_n)$ basis of $V$.  \\

    WTS that $([v_{m + 1}], [v_{m + 2}] \, \hdots [v_n])$ is a basis for $V /W$. \\

    Let $v \in V$ Since $S$ is a basis for $V$
    \begin{align*}
        & v = a_1 w_1 + \hdots a_m w_m + a_{m + 1}v_{m + 1} + \hdots a_n v_n \\
        \implies & [v] = [a_1 w_1 + \hdots a_m w_m] + [a_{m + 1} v_{m + 1} + \hdots a_n w_n] \\ \\
        \implies & [v] = [0] + a_{m + 1}[v_{m + 1}] + \hdots a_n [v_n]
    \end{align*}

    Hence $[v_{m + 1}], \hdots [v_n]$ spans $V /W$. \\


    To show linear independence, let 
    \[
        b_{m + 1} [v_{m + 1}] + \hdots b_n [v_n] = 0
    \]
    for some $b_{m + 1}, \hdots b_n$. \\
    \begin{align*}
        & b_{m + 1} [v_{m + 1}] + \hdots b_n [v_n] = 0 \\
        \implies & \left[ \sum\limits_{i = m+1}^{n} b_i v_i\right] = [0]
    \end{align*}

    That is, 
    \[
        \sum\limits_{i = m+1}^{n} b_i v_i \in W
    \]
    By linear independence of $v_i$'s in $S$, 
    \[
        b_{m + 1} = \hdots = b_n = 0
    \]

    Hence, 
    \begin{align*}
        \dim (V /W) &= \# \{ [v_{m + 1}], \hdots [v_n] \}  \\
        &= n - m \\
        &= \dim V - \dim W
    \end{align*}
\end{proof}

\begin{corollary}
    (New proof of dimension formula for linear maps)  \\
    
    Let $\varphi: V \to W$ be a linear map between $F$-vector spaces. \\
    \[
        \dim V = \dim \ker{\varphi} + \dim Im\varphi
    \]
\end{corollary}
\begin{proof}
    By the homomorphism theorem, 
    \[
        \dim V /\ker{\varphi} \cong Im \varphi
    \]

    Then 
    \begin{align*}
        \dim V / \ker{\varphi} &= \dim Im \varphi \text{ by Homomorphism Theorem}\\
        \dim V \ker{\varphi} &= \dim V - \dim \ker{\varphi} \text{ by above proposition}
    \end{align*}

    Hence
    \[
        \dim V = \dim \ker{\varphi} + \dim Im\varphi
    \]
\end{proof}

\begin{example}
    (Quotient capturing Taylor expansion) \\

    Let $V = C^{\infty}[-1, 1]$ be the space of smooth real-valued functions on $[-1, 1]$ and fix $d \in \mathbb{N}_{\geq 0}$. \\

    Define 
    \[
        W_d = \{ f \in C^{\infty}[-1, 1] \text{ s.t. }  f^{(k)} (0) = 0, k = 0, 1, 2, \hdots d \}  \leq V
    \]


    $W_d$ consists of functions whose Taylor polynomial of degree $d$ at 0 vanishes completely. \\

    Then the quotient 
    \[
        V /W_{d}
    \]
    is naturally isomorphic to the space of polynomials of degree at most $d$. \\

    The isomorphism is induced by the map 
    \[
        \Phi: C^{\infty}[-1, 1] \to \mathcal{P}_d, f \mapsto \Phi(f)(x) = f(0) + f'(0) x + \frac{1}{2!} f''(0) x^2 + \hdots \frac{1}{d!}f^{(d)} (0)  x^d
    \]
    One has 
    \[
        V / W_d = V /\ker{\Phi} \cong Im \Phi = P_d
    \]
\end{example}

\begin{example}
    Recall $V = \mathbb{R}^2, W = span{(1, 0)}$. \\

    Now, we know that 
    \[
        \dim V /W = \dim \mathbb{R}^2 - \dim W = 1
    \]
\end{example}

\subsection{Linear Functionals}
\subsubsection{Dual space}

\begin{definition}
(Linear Functionals) Let $V$ be an $F$-VS. A linear map $f : V \to F$ is alos called a \textbf{linear functional}. 
\end{definition}

\begin{definition}
    Let $F$ be a field and $V$ be a $F$-vector space. The dual space is defined as
    \[
        V^* := Hom_F (V, F)
    \]
    i.e. the vector space of all linear functionals on $V$. 
\end{definition}

\begin{example}
    Examples of linear functionals 
    \begin{itemize}
        \item sum of constants of polynomial
        Let $V = \mathcal{P}_d (\mathbb{R})$, then 
        \[
            f : \mathcal{P}_d ( \mathbb{R}) \to \mathbb{R}, a_0 + a_1 x + \hdots a_d  x^d \mapsto a_0 + a_1 + \hdots a_d
        \]
        \item evaluation map
        Let $V = C^0 [-1, 1]$, then 
        \[
            F_0: C^0 [-1, 1] \to \mathbb{R}, g \mapsto g(0)
        \]
        \item integration map
        \[
            \Phi: C[a, b] \to \mathbb{R}, f \mapsto \int_{a}^{b}   f(x) dx
        \]
        \item linear functional in $F^n$
        Fix $a_1, a_2, \hdots a_n \in F$, define 
        \[
            f : F^n \to F, \begin{pmatrix} v_1 \\ \vdots \\ v_n \end{pmatrix}  \mapsto a_1v_1 + \hdots a_n v_n
        \]
    \end{itemize} 

    Counter examples of linear functionals 
    \begin{itemize}
        \item finding the length 
        \[
        f: \mathbb{R}^3 \to \mathbb{R}, (x, y, z) \mapsto \sqrt{x^2 + y^2 + z^2}
        \]
        is not a linear functional.
        \begin{align*}
            f(-(1, 0, 0)) \neq - f(1, 0, 0)
        \end{align*}
        \item product of coordinates
        \[
            F: \mathbb{R}^2 \to \mathbb{R}, (x, y) \mapsto xy
        \]
        is not a linear functional.
        Take $v_1 = (1, 0), v_2 = (0, 1)$
        \begin{align*}
            F(v_1) = F(v_2) &= 0  \\
            F(v_1) + F(v_2) &= 0 \neq F(v_1 + v_2)= 1
        \end{align*}
    \end{itemize} 
\end{example}

\begin{remark}
    Every linear functional in $F^n$ has the form
    \[
        f : F^n \to F, \begin{pmatrix} v_1 \\ \vdots \\ v_n \end{pmatrix}  \mapsto a_1v_1 + \hdots a_n v_n
    \]
\end{remark}
\begin{proof}
    Let $g \in (F^n)^*$, then 
    \begin{align*}
        g(v) &= g \begin{pmatrix} v_1 \\ v_2 \\ \vdots \\ v_n \end{pmatrix}  = g( v_1 e_1 + \hdots v_n e_n) \\
        &= v_1 g(e_1) + \hdots v_n g(e_n) \text{ by linearity of $g$}. 
    \end{align*}

    if you define $a_i = g(e_i), 1 \leq i \leq n$, then 
    \[
        g(v) = \sum\limits_{i = 1}^{n}  a_i \pi_i
    \]
\end{proof}

\begin{theorem}
    Let $V$ be a vector space over $F$ with basis $S = (s_1, s_2, \hdots s_n)$. Then 
    \begin{enumerate}
        \item $\dim V^* = \dim V$ 
        \item Let $f_i$ be linear map such that 
        \[
            f_i (s_j) = \delta_{ij } = \begin{cases}
                1 \text{ if } j = i \\
                0 \text{ otherwise}
            \end{cases}
        \]
        Then $S^* = (f_1, f_2, \hdots f_n)$ is a basis for $V^*$.
    \end{enumerate} 
\end{theorem}

\begin{remark}
    Recall $\dim W = m, \dim V = n$, 
    \[
        \dim Hom_F(V, W) = mn
    \]
\end{remark}


\begin{proof}

    Proof of (1): 
    \begin{align*}
        \dim V^* &= \dim Hom_F(V, F) \\
        &= \dim V \times \dim F \\
        &= \dim V
    \end{align*}

    Proof of (2): since we know that $\dim V^* = n$, it suffices to show that $S^* = (f_1, f_2, \hdots f_n)$ linearly independent in $V^*$. \\

    We take a linear combination  of $S^*$ that gives the $0$ functional. 
    \[
        a_1 f_1 + a_2 f_2 + \hdots + a_n f_n = 0
    \]
    Apply functionals at $s_j$
    \begin{align*}
        &(a_1 f_1 + a_2 f_2 + \hdots + a_n f_n)(s_j) = 0 (s_j) = 0 \\
        \implies & a_1 f_1 (s_j) + a_2 f_2 (s_j) + \hdots + a_n f_n (s_j) = 0 \\
        \implies & a_j f_j s_j = 0 \\
        \implies & a_j = 0
    \end{align*}
    This is true for all $1 \leq j \leq n$, therefore $S^* = (f_1, f_2, \hdots f_n)$ linearly independent.
\end{proof}

\begin{definition}
    $S^* = (f_1, f_2, \hdots f_n)$ from theorem above is called the dual basis of $S$. 

    Each $f_i$ is denoted 
    \[
        f_i = S_i^*
    \]
    
\end{definition}

\begin{example}
    Let $V = F^n$, and $S = (e_1, e_2, \hdots e_n)$ is the standard basis where $e_i = (0, 0, \hdots 1, \hdots, 0)^T$  (only nonzero element is 1 at the $i$-th position). \\

    Then 
    \begin{align*}
        e_i^* \begin{pmatrix} v_1 \\ v_2 \\ \vdots \\ v_n \end{pmatrix}  &= e_i^* \left( \sum\limits_{j= 1}^{n} v_j e_j  \right) \\
        &= \sum\limits_{j = 1}^{n}  v_j e_i^*(e_j ) \\
        &= v_i e_i^* (e_i) \\
        &= v_i
    \end{align*}
\end{example}

\subsubsection{Duality Theorem}

\begin{definition}
    Since $V^*$ is again a vector space over $F$. Define the bidual space as 
    \[
        V^{**} := (V^*)^* = Hom_F (V^*, F)
    \]
\end{definition}

\begin{remark}      
    If $\dim V < \infty$, 
    \[
        \dim (V^{**}) = \dim V^* = \dim V
    \]
\end{remark}        

\begin{theorem}
    Let $V$ be a finite-dimensional $F$-vector space. Then, there exists a natural isomorphism 
    \[
        \Theta: V \to V^{**} = Hom_F(V^*, F), v \mapsto \theta(v) = \theta_v
    \]
    Where 
    \[
        \theta_v(f) = f(v) \text{ for all } f \in V^*
    \]
    i.e. $\theta_v$ is an evaluation functional (taking linear functionals to scalars).
\end{theorem}

%TODO: fix all function definitions, the domain and codomain would ideally be on the top line, and the mapping should come in the second line.
















