\section{Class 12}

\begin{definition}
    (rank) The (column) rank of $A \in \mathbb{R}^{m \times n}$ is the maximal number of linearly independent columns, i.e. the dimension of the space spanned by column vectors in $\mathbb{R}^m$. \\

    The row rank of $A$ is defined as the number of linearly independent rows, i.e. the dimension of the space spanned by row vectors in $\mathbb{R}^n$.
\end{definition}

\begin{remark}
    Let $A \in F^{m \times n}$, $V, W$ $F$-vector spaces with $S = (s_1, s_2, \hdots s_n)$ asis for $V$, $T = (t_1, t_2, \hdots t_m)$ basis for $W$. \\

    Let $ \phi: V \to W$ linear such that 
    \[
        \left[ \phi \right]_{S \to T} = A
    \]

    \begin{align*}
        \dim \left( Im( \phi) \right)  &= \dim \left( span(\phi(s_1), \phi(s_2), \hdots \phi(s_n)) \right)  \\
        &= \dim span( \text{columns of } A) \\
        &= \rank(A)
    \end{align*}
    One has 
    \[
        \rank \left( \phi \right)  = \rank(A) 
    \]
\end{remark}

\begin{corollary}
    If $A, B$ are equivalent matrices, 
    \[
        \rank(A) = \rank(B)
    \]
\end{corollary}
\begin{proof} To be updated
\end{proof}

\begin{theorem}     
    Every $A \in Mat_{m \times n}(F)$ is equivalent to exactly one matrix of the form 
\[ 
\begin{bmatrix}
    1 & 0 & 0 & \hdots & 0 & 0 & \hdots & 0 \\
    0 & 1 & 0 & \hdots & 0 & 0 & \hdots & 0 \\
    0 & 0 & 1 & \hdots & 0 & 0 & \hdots & 0 \\
    \vdots & \vdots & \vdots & \ddots & \vdots & \vdots & \vdots & \vdots \\
    0 & 0 & 0 & \hdots & 1 & 0 & \hdots & 0 \\
    0 & 0 & 0 & \hdots & 0 & 0 & \hdots & 0 \\
    0 & 0 & 0 & \hdots & 0 & 0 & \hdots & 0 \\
\end{bmatrix} = \left[
    \begin{array}{c | c}
    I_r & 0 \\
    \hline
    0 & 0
    \end{array}
\right]
\]
where $r = \rank(A)$, and this form is known as the \textbf{rank-normal} form.
\end{theorem}       

\begin{proof}       
    Let $B^n$ be the standard basis for $\mathbb{R}^n$, $B^m$ be the standard basis for $\mathbb{R}^m$. 

    Consider $\phi: \mathbb{R}^n \to \mathbb{R}^m$ such that 
    \[
        \left[ \phi \right]_{B^n \to B^m} = A
    \]
    Let $S_2$ be a basis for $\ker( \phi)$. Extend $S_2$ to be a basis for $\mathbb{R}^n$. \\

    \[
        \tilde{S} = \{s_1, s_2, \hdots s_r, s_{r + 1}, \hdots s_n \} 
    \]
    
\end{proof}     








\newpage