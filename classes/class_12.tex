\section{Class 12}

\subsection{Equivalence and rank of matrices}

\begin{definition}
    (Equivalent matrices) $A, B \in F^{m \times n}$. $B$ is equivalent to $A$ if there are matrices $c \in Gl_m(F), D \in Gl_n(F)$ so that 
    \[
        B = C \cdot A \cdot D
    \]
\end{definition}

\begin{definition}
    (Similar matrices) $A, B \in F^{m \times n}$. $B$ is similar to $A$ if $m = n$ and 
    \[
        B = C^{-1} A C
    \]
    for some $C \in Gl_n(F)$
\end{definition}

\begin{remark}
    Recall that if $\phi: V \to W$ linear, and $S, \tilde{S}$ basis for $V$, $T, \tilde{T}$ basis for $W$, then 
    \[
        \left[ \phi \right]_{\tilde{S} \to \tilde{T}} = C_{T \to \tilde{T}} \left[ \phi \right]_{S \to T} C_{\tilde{S} \to S}
    \]

    Matrices that represent the same linear map with respect to different basis are equivalent.  \\
\end{remark}

\begin{remark}
    Reciprocally, given equivalent matrices $A, B \in F^{m \times n}$, and $V, W$ vector spaces of dimension $n, m$ with bases $B_v, B_w$, the matrix $A$ represents a unique transformation 
    \begin{align*}
        \phi: V & \to W \\
        \left[ \phi \right]_{B_V \to B_W} = A
    \end{align*}
    Then $B$ represents $\phi$ with respect to some new bases.
\end{remark}

\begin{remark}
    In the case $ \phi: V \to V$, $S, \tilde{S}$ bases for $V$, 
    \[
        \left[ \phi \right]_{\tilde{S} \to \tilde{S}} 
        = C_{S \to \tilde{S}} \left[ \phi \right]_{S \to S} C_{\tilde{S} \to S} 
        = C_{\tilde{S} \to S}^{-1} \left[ \phi \right]_{S \to S} C_{\tilde{S} \to S}
    \]
    and in this case 
    \[
        \left[ \phi \right]_{\tilde{S} \to \tilde{S}}
    \]
    is similar to
    \[
        \left[ \phi \right]_{{S} \to {S}}
    \]
\end{remark}

\begin{remark}
    \textbf{Equivalent matrices} represent the same linear map with different bases in the domain and the target.  \\
    
    \textbf{Similar matrices} represent the same endomorphism with respect to different bases. 
\end{remark}

\begin{definition}
    (Equivalence relation) Let $X$ be a set. A relation $\sim$ on $X$ is called an \textbf{equivalence relation} if it is 
    \begin{enumerate}
        \item symmetric 
        \[
            x \sim y \iff y \sim x \text{ for all } x, y \in X
        \]
        \item reflexive
        \[
            x \sim x \text{ for all } x \in X
        \]
        \item transitive
        \[
        x \sim y, y \sim z \implies x \sim z \text{ for all }x, y, z, \in X
        \]
    \end{enumerate}
\end{definition}

\begin{remark}
    "Equivalence" is an equivalence relation on $F^{m \times n}$. \\

    "Similarity" is an equivalnece relation on $F^{n \times n}$. 
    \begin{enumerate}
        \item symmetry
        \[
            B = C^{-1} AC \implies CBC^{-1} = A
        \]
        \item reflexivity
        \[
            A = I_n^{-1} A I_n
        \]
        \item transitivity
        \[
            B = C^{-1}AC, D = \tilde{C}^{-1} B \tilde{C}
        \] for some $C, \tilde{C} \in Gl_n(F)$, then 
        \[
        D = \tilde{C}^{-1} (C^{-1}AC)\tilde{C} = (C \tilde{C})^{-1} A (C \tilde{C})
        \]
        and $C\tilde{C} \in Gl_n(F)$
    \end{enumerate}
\end{remark}

\begin{definition}
    (rank) The (column) rank of $A \in \mathbb{R}^{m \times n}$ is the maximal number of linearly independent columns, i.e. the dimension of the space spanned by column vectors in $\mathbb{R}^m$. \\

    The row rank of $A$ is defined as the number of linearly independent rows, i.e. the dimension of the space spanned by row vectors in $\mathbb{R}^n$.
\end{definition}

\begin{remark}
    Let $A \in F^{m \times n}$, $V, W$ $F$-vector spaces with $S = (s_1, s_2, \hdots s_n)$ asis for $V$, $T = (t_1, t_2, \hdots t_m)$ basis for $W$. \\

    Let $ \phi: V \to W$ linear such that 
    \[
        \left[ \phi \right]_{S \to T} = A
    \]

    \begin{align*}
        \dim \left( Im( \phi) \right)  &= \dim \left( span(\phi(s_1), \phi(s_2), \hdots \phi(s_n)) \right)  \\
        &= \dim span( \text{columns of } A) \\
        &= \rank(A)
    \end{align*}
    One has 
    \[
        \rank \left( \phi \right)  = \rank(A) 
    \]
\end{remark}

\begin{corollary}
    If $A, B$ are equivalent matrices, 
    \[
        \rank(A) = \rank(B)
    \]
\end{corollary}
\begin{proof} To be updated
\end{proof}

\begin{theorem}     
    Every $A \in Mat_{m \times n}(F)$ is equivalent to exactly one matrix of the form 
\[ 
\begin{bmatrix}
    1 & 0 & 0 & \hdots & 0 & 0 & \hdots & 0 \\
    0 & 1 & 0 & \hdots & 0 & 0 & \hdots & 0 \\
    0 & 0 & 1 & \hdots & 0 & 0 & \hdots & 0 \\
    \vdots & \vdots & \vdots & \ddots & \vdots & \vdots & \vdots & \vdots \\
    0 & 0 & 0 & \hdots & 1 & 0 & \hdots & 0 \\
    0 & 0 & 0 & \hdots & 0 & 0 & \hdots & 0 \\
    0 & 0 & 0 & \hdots & 0 & 0 & \hdots & 0 \\
\end{bmatrix} = \left[
    \begin{array}{c | c}
    I_r & 0 \\
    \hline
    0 & 0
    \end{array}
\right]
\]
where $r = \rank(A)$, and this form is known as the \textbf{rank-normal} form.
\end{theorem}       

\begin{proof}       
    Let $B^n$ be the standard basis for $\mathbb{R}^n$, $B^m$ be the standard basis for $\mathbb{R}^m$. 

    Consider $\phi: \mathbb{R}^n \to \mathbb{R}^m$ such that 
    \[
        \left[ \phi \right]_{B^n \to B^m} = A
    \]

    We know that such a map exists because 
    \begin{align*}
        Hom \left( \mathbb{R}^n , \mathbb{R}^m \right)  &\to F^{m \times n} \\
        \psi &\mapsto \left[ \psi \right]_{B_n \to B_m}
    \end{align*}
    is isomorphism, and therefore surjective. 

    Let $S_2$ be a basis for $\ker( \phi)$. Extend $S_2$ to be a basis for $\mathbb{R}^n$. \\

    \[
        \tilde{S} = \{\underbrace{s_1, s_2, \hdots s_r}_{S^1}, \underbrace{s_{r + 1}, \hdots s_n}_{S^2} \} 
    \]

    where $r$ is such that $\dim \ker( \phi) = n - r$. \\

    Since $\phi(s_i) = 0$ for all $r < i \leq n$, and $\phi(s_1), \phi(s_2) \hdots \phi(s_r)$ linearly independent, we know that 
    \[
        \left( \phi(s_1), \phi(s_2), \hdots \phi(s_r) \right) 
    \]
    is a basis for $Im( \phi) \leq F^m$. \\

    For $1 \leq i \leq r$, we define 
    \[
        t_i := \phi(s_i)
    \]
    Hence 
    \[
        T = (t_1, t_2, \hdots t_r) 
    \]
    is a basis for $Im( \phi)$. \\

    Extend $T$ to basis $\tilde{T}$ for $F^m$.
    \[
        \tilde{T} = (t_1, t_2, \hdots, t_r, t_{r + 1}, \hdots, t_m)
    \]

    Note that $[ \phi]_{\tilde{S} \to \tilde{T}}$ is in rank normal form, and 
    \[
        \left[ \phi \right]_{\tilde{S} \to \tilde{T}} = C_{B^m \to \tilde{T}} \underbrace{\left[ \phi \right]_{B^n \to B^m}}_{=A} C_{\tilde{S} \to B^n}
    \]

    Hence $[\phi]_{\tilde{S} \to \tilde{T}}$ and $A$ equivalent. \\

    Since the number of nonzero rows is exactly $rank(A)$, this is unique. 
    
\end{proof}     

\begin{remark}
    Elementary matrices are those obtained by performing a single elementary row operation on identity. \\

    Multiplying a matrix on the left by an elementary matrix applies the corresponding row operation on $A$. 
\end{remark}

\begin{remark}
    Multiplying $A$ on the right by an elementary matrix performs an analogous column operation.
\end{remark}

\begin{example}
    Let $A$
    \[
        A = \begin{pmatrix} 
          1 & 2 \\ 3 & 4  
        \end{pmatrix}
    \]
    Let $B$ be an elementary matrix reprensting the row operation $R_2 + 3R_1 \to R_2$
    \[
        \begin{pmatrix} 
          1 & 0 \\ 3 & 1  
        \end{pmatrix}
    \]

    Then,  multiplying on the left performs $\leftarrow R_2 + 3 R_1 \to R_2$
    \[
        \begin{pmatrix} 
          1 & 0 \\ 3 & 1  
        \end{pmatrix} \begin{pmatrix} 
          1 & 2 \\ 3 & 4  
        \end{pmatrix} = \begin{pmatrix} 
          1 & 2 \\ 6 & 10  
        \end{pmatrix}
    \]
    Multiplying on the right performs $C_1 + 3 C_2 \to C_1$
\end{example}

\begin{proposition}
    The rank of a matrix does not change under elementary row or column operations.
\end{proposition}

\begin{proof}
    Let $A \in F^{m \times n}$ and $E \in Mat_m(F)$ corresponding to an elementary row operation. Then 
    \[
        B = EA = EA I
    \]

    Hence $B$ and $A$ are equivalent, and 
    \[
        \rank(B) = \rank(A)
    \]

    Similarly for column operations.
\end{proof}

\begin{example}
    Let $A$ be $ \begin{pmatrix} 
      1 & 2 & 3 \\  
      4 & 5 & 6 \\  
      7 & 8 & 9 \\  
    \end{pmatrix}
    $. Then 
    \begin{align*}
        \begin{pmatrix} 
            1 & 2 & 3 \\
            4 & 5 & 6 \\  
            7 & 8 & 9 \\  
        \end{pmatrix}
        \overset{\text{row operations}}{\longrightarrow} 
        \begin{pmatrix} 
            1 & 2 & 3 \\
            0 & -3 & -6 \\  
            0 & -6 & -12 \\  
        \end{pmatrix}
        \overset{\text{col operations}}{\longrightarrow} 
        \begin{pmatrix} 
            1 & 0 & 0 \\
            0 & -3 & -6 \\  
            0 & -6 & -12 \\  
        \end{pmatrix} \\
        \overset{\text{row operations}}{\longrightarrow} 
        \begin{pmatrix} 
            1 & 0 & 0 \\
            0 & 1 & 2 \\  
            0 & 1 & 2 \\  
        \end{pmatrix} 
        \overset{\text{row operations}}{\longrightarrow} 
        \begin{pmatrix} 
            1 & 0 & 0 \\
            0 & 1 & 2 \\  
            0 & 0 & 0 \\  
        \end{pmatrix} 
        \overset{\text{col operations}}{\longrightarrow} 
        \begin{pmatrix} 
            1 & 0 & 0 \\
            0 & 1 & 0 \\  
            0 & 0 & 0 \\  
        \end{pmatrix} 
    \end{align*}

    Hence $\rank(A) = 2$.
\end{example}

\begin{remark}
    We can use row and column operations to put a matrix in \textbf{rank normal form}, where the rank is easy to observe.
\end{remark}

\begin{remark}
    Row rank and (column) rank agreee.
\end{remark}

\begin{remark}
    Equivalence of matrices reserves row rank. \\

    Let $A \in F^{m \times n}$  and $B$ be its rank normal form. Then, 
    \[
        \text{row rank} (A) = \rank (A) = \text{row rank} (B) = \rank(B) = r
    \]
\end{remark}

\subsection{Systems of linear equations}

\begin{theorem}
    Let $A \in F^{m \times n}, b \in F^m$. And $Ax = b$ be a system of linear equations. \\

    Then 
    \begin{enumerate}
        \item The system is solvable if and only if 
        \[
            \rank(A) = \rank(A | b)
        \]
        \item If $b = 0$ then the solution space of $Ax = 0$ is a subspace of $F^n$ of dimension $n - \rank(A)$
        \item Let $x_0$ be a solution of $Ax = b$. Then every solution of $Ax = b$ has the form 
        \[
            x = x_0 + y
        \]
        where $y$ is a solution of the homogenous system. 
    \end{enumerate} 
\end{theorem}
\begin{proof} \\

\textbf{Proof of (1)}: Let $V = F^n, W = F^m$, with standard basis $\beta^n, \beta^m$, there exits unique 
\[
    \phi: V \to W
\]
so that 
\[
    [\phi]_{\beta^n \to \beta^m}  = A
\]

Then $Ax = b$ is the coordinate representation of $\phi (x) = b$. 
\begin{align*}
    \text{system is solvable} & \iff \phi(x) = b \text{ has a solution} \\
    & \iff b \in Im( \phi) = span \{ \text{ columns of } A \}  \\
    & \iff b \text{ is linear combination of columns of } A \\
    & \iff \rank(A) = \rank(A | b)
\end{align*}

\textbf{Proof of (2)}: Again let $ \phi: F^n \to F^m$ such that 
\[
    \left[ \phi \right]_{\beta^n \to \beta^m} = A
\]

By dimension formula, 
\[
    n = \dim Im( \phi) + \dim \ker( \phi)
\]

Where $\dim Im( \phi) = \rank(A)$. 
Hence 
\[
    \dim \ker( \phi) = n - \rank(A)
\]

\textbf{Proof of (3)}: Suppose $Ax_0 = b$. Then 
\begin{align*}
    Ax = b & \iff Ax = Ax_0 \\
    & \iff A(x - x_0) = 0 \\
    &\iff x - x_0 \in \ker(A) \\
    & \iff x = x_0 + y, y \in \ker(A)
\end{align*}





\end{proof}





















\newpage